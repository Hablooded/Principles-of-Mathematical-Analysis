\documentclass{article}
\input{config}
\DeclareMathOperator{\Ker}{Ker}
\DeclareMathOperator{\Ima}{Im}
\DeclareMathOperator{\Aut}{Aut}
\DeclareMathOperator{\ch}{ch}
\DeclareMathOperator{\Gal}{Gal}

\newcommand{\ra}{\rightarrow}
\newcommand{\ds}{\displaystyle}
\newcommand{\mt}{\mapsto}
\newcommand{\xra}{\xrightarrow}

\newcommand{\CC}{\mathbb{C}}
\newcommand{\QQ}{\mathbb{Q}}
\newcommand{\RR}{\mathbb{R}}
\newcommand{\ZZ}{\mathbb{Z}}
\newcommand{\FF}{\mathbb{F}}


\title{\Huge \textbf{Chapter 5}\\
Defferentiation}
\begin{document}
\maketitle
\section{实值函数的微分}
\begin{definition}
    设\(f\)是定义在\([a,b]\)上的实值函数。对于任意的\(x\in[a,b]\),定义\[
f'(x) =\lim_{t\ra x} \frac{f(t)-f(x)}{t-x} \quad(a<t<b,t\neq x)
\] \(f'\)称为\(f\)的导数。
\end{definition}

\begin{theorem}
    设\(f\)是定义在\([a,b]\)的函数。若\(f\)在\(x\in[a,b]\)处可导,则\(f\)在\(x\)处连续。
\end{theorem}
这个定理的逆命题是不成立的。事实上,存在处处不可导的连续函数,

\begin{theorem}
    这是关于导数的运算,不多赘述。
\end{theorem}
\begin{theorem}
    设\(f\)在\([a,b]\)上连续,在某些点\(x\in[a,b]\)上存在\(f'(x)\),\(g\)定义在包含\(f\)的值域的区间\(I\)上,且\(g\)在点\(f(x)\)上可导。若\[
    h(t)=g(f(t))\quad(a\leq t\leq b)
    \]则\(h\)在\(x\)处可导,且\[
    h'(x)=g'(f(x))f'(x)
    \]
\end{theorem}

\section{中值定理}

\begin{definition}
    设\(f\)是定义在度规空间\(X\)上的实值函数。我们说\(f\)在\(p\in X\)有局部最大值,如果存在\(\delta>0\),使得对于任意的\(q\in X\),\(d(p,q)<\delta\),都有\(f(q)\leq f(p)\)。局部最大值的定义是类似的。
\end{definition}

\begin{theorem}
    设\(f\)定义在\([a,b]\)上,若\(f\)在\(x\in[a,b]\)上有局部最大值/最小值,且若\(f'(x)\)存在,则\(f'(x)=0\)。
\end{theorem}

\begin{theorem}
    若\(f,g\)是\([a,b]\)上的连续实值函数,在\((a,b)\)上可微,则存在点\(x\in(a,b)\)使得\[
    [f(b)-f(a)]g'(x) =[g(b)-g(a)]f'(x)
    \]注意到端点处不要求可导。
\end{theorem}
我们有上述一般化的中值定理的常用形式。
\begin{theorem}
    若\(f\)是\([a,b]\)上的实值连续函数,在\((a,b)\)上可微,则存在\(x\in(a,b)\),使得\[
    f(b)-f(a)=(b-a)f'(x)
    \]
\end{theorem}

\begin{theorem}
    设\(f\)在\((a,b)\)上可微。\begin{enumerate}[label=(\alph*)]
        \item 若对于任意的\(x\in(a,b)\),\(f'(x)\geq0\),则\(f\)单调递增。
        \item 若对于任意的\(x\in(a,b)\),\(f'(x)=0\),则\(f\)是常函数。
        \item 若对于任意的\(x\in(a,b)\),\(f'(x)\leq0\),则\(f\)单调递减。
    \end{enumerate}
\end{theorem}
\begin{proof}
    利用中值定理是容易证明的。
\end{proof}


\section{导函数的连续性}

\begin{theorem}
    设\(f\)是在\([a,b]\)上可导的实值函数,假设\(f'(a)<\lambda <f'(b)\),则存在点\(x\in[a,b]\),使得\(f'(x)=\lambda\)。\par
    若\(f'(a)>f'(b)\),也有类似的结果成立。
\end{theorem}

\begin{corollary}
    若\(f\)在\([a,b]\)是可导,则\(f'\)在\([a,b]\)上不能有第一类不连续点。
\end{corollary}

\section{L'HOSPITAL 法则}

\begin{theorem}
    设\(f,g\)是\((a,b)\)上的可导实值函数。对于任意的\(x\in(a,b)\),\(g'(x)\neq 0\)。设\[
    \frac{f'(x)}{g'(x)}\ra A\quad(x\ra a)
    \]
    若\[
    f(x)\ra 0,g(x)\ra 0\quad(x\ra a)
    \]或者
    \[g(x) \ra +\infty \quad(x\ra a) \]
    则有\[
    \frac{f(x)}{g(x)}\ra A\quad(x\ra a)
    \]
\end{theorem}

\section{高阶导数}
\section{向量值函数的导数}

\end{document}