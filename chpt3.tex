\documentclass{article}
\input{config}
\DeclareMathOperator{\Ker}{Ker}
\DeclareMathOperator{\Ima}{Im}
\DeclareMathOperator{\Aut}{Aut}
\DeclareMathOperator{\ch}{ch}
\DeclareMathOperator{\Gal}{Gal}

\newcommand{\ra}{\rightarrow}
\newcommand{\ds}{\displaystyle}
\newcommand{\mt}{\mapsto}
\newcommand{\xra}{\xrightarrow}

\newcommand{\CC}{\mathbb{C}}
\newcommand{\QQ}{\mathbb{Q}}
\newcommand{\RR}{\mathbb{R}}
\newcommand{\ZZ}{\mathbb{Z}}
\newcommand{\FF}{\mathbb{F}}


\title{
\Huge \textbf{Chapter 3}
\\
Sequences and Serie}
\author{}
\date{}

\begin{document}

\maketitle

\section{部分和}
\begin{theorem}
    考虑数列\(\{a_n\},\{b_n\}\),设\[
    A_n = \sum_{i=1} ^n  a_i \quad(n>0)
    \]约定\(A_{-1} =0\),若\(0\leq p\leq  q\),则我们有\[
    \sum_{n=p}^q a_nb_n =\sum_{n=p}^{q-1}A_n (b_n-b_{n-1})+A_qb_q-A_{p-1}b_p
    \]
\end{theorem}

\begin{theorem}
    假设\begin{enumerate}[label=(\arabic*)]
        \item \(\sum a_n\)的部分和\(A_n\)构成一个有界数列;
        \item \(b_0\geq b_1\geq b_2\geq \cdots \);
        \item \(\ds \lim _{n\ra \infty} b_n =0\).
    \end{enumerate}
    则\(\sum a_nb_n\)收敛。
\end{theorem}

\begin{theorem}
    假设\begin{enumerate}[label=(\arabic*)]
        \item \(|c_1|\geq |c_2|\geq |c_3|\geq \cdots\);
        \item \(c_{2m-1}\geq 0,c_{2m}\leq 0\quad(m=1,2,3,....)\);
        \item \(\ds \lim_{n\ra \infty }c_n = 0\).
    \end{enumerate}
    则\(c_n \)收敛。
\end{theorem}

特别地,满足条件{\it (2)}的数列称为交错列。

\begin{theorem}
    假设幂级数\(\sum c_n z^n\)的收敛半径是\(1\),且\(c_0\geq c_1\geq c_2\geq\cdots\),\(\ds\lim_{n\ra \infty}c_n=0\),则\(\sum c_nz^n\)在\(|z|=1\)上除了\(z=1\)外任意一点收敛。\(z=1\)时的收敛性不确定。
\end{theorem}

\section{绝对收敛}
\begin{definition}
    数列\(\{a_n\}\)满足\( \sum |a_n|\)收敛,则称\(\{a_n\}\)绝对收敛。
\end{definition}
\section{数列的加法和乘法}
数列的加法是平凡的。数列的乘法是根据幂级数的乘法定义的。

\section{重排}
\begin{theorem}
    对于收敛的级数,它的任意重排是有界的。
\end{theorem}

\begin{theorem}
    对于绝对收敛的级数,它的任意重排均收敛且收敛于原数列的收敛值。
\end{theorem}

\end{document}
