\documentclass{article}
\input{config}
\usepackage{bm}
\DeclareMathOperator{\Ker}{Ker}
\DeclareMathOperator{\Ima}{Im}
\DeclareMathOperator{\Aut}{Aut}
\DeclareMathOperator{\ch}{ch}
\DeclareMathOperator{\Gal}{Gal}

\newcommand{\ra}{\rightarrow}
\newcommand{\ds}{\displaystyle}
\newcommand{\mt}{\mapsto}
\newcommand{\xra}{\xrightarrow}

\newcommand{\CC}{\mathbb{C}}
\newcommand{\QQ}{\mathbb{Q}}
\newcommand{\RR}{\mathbb{R}}
\newcommand{\ZZ}{\mathbb{Z}}
\newcommand{\FF}{\mathbb{F}}
\newcommand{\ve}{\varepsilon}

\title{ {\Huge  \textbf{Chapter 4}} \\Continuity}
\date{}
\begin{document}
\maketitle

\section{函数的极限}
\begin{definition}
    设\(X,Y\)是度量空间,\(E\subset X\),\(f:E\to Y\),\(p\)是\(E\)的极限点。若对于任意的\(\varepsilon>0\),存在\(\delta>0\),使得对于任意的\(x\in E\),且\[0<d_X(x,p)<\delta\]都有\[
    d_Y(f(x),q)<\ve
    \]则称\[
    \lim_{x\ra p}f(x) =q
    \]
\end{definition}

\begin{theorem}
    记号继承定义1.1。设\(E\)中的序列\(\{p_n\}\)满足\[
    p_n\neq p\quad \lim_{n\ra \infty }p_n =p
    \] 则\[
    \lim_{x\ra p}f(x) =q
    \]当且仅当对于任意\(\{p_n\}\)都有\[
    \lim_{n\ra \infty}f(p_n) =q.
    \]
\end{theorem}

\begin{proof}
    充分性是显见的;必要性,用反证法容易得到矛盾。
\end{proof}
\begin{definition}
    对函数定义\(f+g,f\cdot g\),它的具体内容由像空间的运算确定。
\end{definition}
\begin{corollary}
    如果\(f\)在\(p\)有极限,则这个极限是唯一的。
\end{corollary}


\begin{theorem}
    \(X\)是度量空间,设\(E\subset X\),\(p\)是\(E\)的极限点,\(f,g\)是\(E\)上的复函数,且\[
    \lim_{x\ra p} f(x) =A,\quad \lim_{x\ra p}g(x) =B
    \]则极限运算在这里构成域同态。
\end{theorem}

\section{连续函数}
\begin{definition}
    设\(X,Y\)是度量空间,\(E\subset X\),\(p\in E\),\(f:E\to Y\)。若对于任意的\(\ve>0\),存在\(\delta>0\),使得对于任意的\(x\in E\),\(0<d_X(x,p)<\delta\),都有\[
    d_Y(f(x),f(p))<\ve
    \]则称\(f\)在\(p\)点连续。
\end{definition}

\begin{theorem}
    记号同定义2.1,并假设\(p\)也是\(E\)的极限点(即\(p\)不是\(E\)的孤立点),则\(f\)在\(p\)点是连续的当且仅当\(\ds \lim_{x\ra p} =f(p)\)。
\end{theorem}

\begin{theorem}
    设\(X,Y,Z\)是度量空间,\(E\subset X\),\(f:E\ra Y,g:f(E)\ra Y\),定义\(h:E\ra Z\)为\[
    h(x)=f(g(x))\quad(x\in E)
    \] 
    若\(f\)在\(p\)处连续,\(g\)在\(f(p)\)处连续,则\(h\)在\(p\)处连续。
\end{theorem}

\begin{theorem}
    设\(X,Y\)是度量空间。映射\(f:X\ra Y\)在\(X\)上是连续的,当且仅当对于任意的开集\(V\subset Y\),\(f^{-1}(V)\)在\(X\)中都是开的。
\end{theorem}

\begin{corollary}
    延续记号。映射\(f:X\ra Y\)在\(X\)上是连续的,当且仅当对于任意的闭集\(C\subset Y\),\(f^{-1}(C)\)是闭的。
\end{corollary}


\begin{theorem}
    设\(f,g\)是度量空间\(X\)上的连续复函数,则\(f+g,f\cdot g,f/g\)在\(X\)上是连续的。
\end{theorem}


\begin{theorem}
    (a) 设\(f_1,...,f_k\)是度量空间\(X\)上的实值函数,\(\bm f:X\ra \RR^k\)定义为\[
    f(x) =(f_1(x),...,f_k(x))\quad(x\in X)
    \] 则\(\bm f\)连续当且仅当\(f_1,...,f_k\)连续。

    (b) 若\(\bm{f,g}:X\ra \RR^k\)是连续的,则\(\bm{f+g,f\cdot g}\)也是连续的。
\end{theorem}

\begin{example}
        若\(x_1,...,x_k\)是\(\bm x \in \RR^k\)的坐标,定义函数\(\phi_i\)为\[
    \phi_i(\bm x) =x_i 
    \]这是一个连续函数。
\end{example}
\section{连续和紧性}
\begin{definition}
    映射\(\bm f:E \ra \RR^k\)是有界的,如果存在实数\(M\)使得对于任意的\(x\in E\),\(|f(x)|\leq M\)。
\end{definition}

\begin{theorem}
    设\(f\)是紧度量空间\(X\)到度量空间\(Y\)的连续映射,则\(f(X)\)是连续的。
\end{theorem}

\begin{theorem}
    设\(\bm f\)是紧度量空间\(X\)到\(\RR^k\)的连续映射,则\(\bm f(X)\)是闭的且是有界的。因此,\(\bm f\)是有界的。
\end{theorem}

\begin{theorem}
    设\(f\)是紧度量空间\(X\)上的连续实值函数,且\[
    M =\sup_{x\in X}f(x)\quad m=\inf_{x\in X}f(x)
    \]则存在点\(p,q\in X\),使得\(f(p)=M,f(q)=N\)。
\end{theorem}

\begin{theorem}
    设\(f\)是紧度量空间\(X\)到度量空间\(Y\)一一对应的连续映射,则逆映射\(f^{-1}:Y\ra X\)也是连续映射。
\end{theorem}

\begin{definition}
    设\(X,Y\)是度规空间,\(f:X\ra Y\),若对于任意的\(\ve>0\),存在\(\delta>0\)使得对于任意的\(p,q\in X\),若\(d_X(p,q)<\delta\),则有\[
    d_Y(f(p),f(q)) <\ve
    \]
\end{definition}

\begin{theorem}
    设\(f\)是紧度量空间\(X\)到度规空间\(Y\)的连续映射,则\(f\)在\(X\)上一致连续。
\end{theorem}

\begin{theorem}
    设\(E\)在\(\RR\)上不是紧集,则\\
    (a) \(E\)上存在无界的连续函数。\\
    (b) \(E\)上存在连续有界,但没有最大值的函数。\\
    进一步地,若\(E\)有界,则\\
    (c) \(E\)上存在不一致连续的连续函数。
\end{theorem}

\section{连续和连通性}
\begin{theorem}
    设\(X,Y\)是度规空间,\(f:X\ra Y\)是连续映射。若\(E\)是\(X\)的连通子集,则\(f(E)\)也是连通的。
\end{theorem}

\begin{theorem}
    设\(f\)是区间\([a,b]\)上的实值连续函数。若\(f(a)<f(b)\),实数\(c\)满足\(f(a)<c<f(b)\),则存在点\(x\in (a,b)\)使得\(f(x) =c\)。
\end{theorem}


\section{不连续性}
\begin{definition}
    设\(f\)定义在\((a,b)\)上。考虑点\(x\in[a.b)\),则记\[
    f(x+)=q
    \]若对于任意的\((x,b)\)中收敛于\(x\)的数列\(\{t_n\}\),当\(n\ra\infty\)时,都有\(f(t_n)\ra q\)。类似地,将\(\{t_n\}\)限制在\((a,x)\),可以定义\(f(x-)\)。
\end{definition}

容易知道\(\ds\lim_{t\ra x}f(t)\)当且仅当\(f(x+)=f(x-)=\ds\lim_{t\ra x}f(t)\)。
\begin{definition}
    设\(f\)定义在\((a,b)\)。若\(f\)在点\(x\)处不连续,且\(f(x+),f(x-)\)存在,则称\(f\)在点\(x\)处有第一类不连续。
\end{definition}
\section{单调函数}


\end{document}