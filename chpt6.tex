\documentclass{article}
\input{config}

% 定义带短下划线的积分符号
\newcommand{\uli}{\int\limits_{\bar{\ }}}

\DeclareMathOperator{\Ker}{Ker}
\DeclareMathOperator{\Ima}{Im}
\DeclareMathOperator{\Aut}{Aut}
\DeclareMathOperator{\ch}{ch}
\DeclareMathOperator{\Gal}{Gal}

\newcommand{\ra}{\rightarrow}
\newcommand{\ds}{\displaystyle}
\newcommand{\mt}{\mapsto}
\newcommand{\xra}{\xrightarrow}

\newcommand{\CC}{\mathbb{C}}
\newcommand{\QQ}{\mathbb{Q}}
\newcommand{\RR}{\mathbb{R}}
\newcommand{\ZZ}{\mathbb{Z}}
\newcommand{\FF}{\mathbb{F}}
\newcommand{\ol}{\overline}
\newcommand{\sR}{\mathscr R}

\title{\Huge \textbf{Chapter 6}\\
RIEMANN-STIELTJES 积分}

\date{}
\begin{document}
\maketitle
\section{积分的定义和存在性}

\begin{definition}
    给定区间\([a,b]\)。我们说\([a,b]\)的一个划分\(P\)是有限点集\(x_0,x_1,...,x_n\),其中\[
    a=x_0\leq x_1\leq \cdots\leq x_{n-1}\leq x_n=b
    \]记\[\Delta x_i=x_i-x_{i-1} \quad(i=1,...,n)\]
    设\(f\)是定义在\([a,b]\)上的有界实函数。与特定的划分\(P\)相对应,设\begin{gather*}
        M_i=\sup f(x) \quad(x_{i-1}\leq x\leq x_i)\\
        m_i =\inf f(x)\quad(x_{i-1}\leq x\leq x_i)\\
        U(P,f)=\sum_{i=1}^n M_i\Delta x_i\\
        L(P,f)=\sum_{i=1}^nm_i\Delta x_i
    \end{gather*}
    进一步地,考虑所有的\(P\),定义\begin{gather*} 
\overline{\int_a^b} f \, dx=   \inf U(P,f)\\
\underline{\int_a^b} f \, dx=\sup L(P,f)
    \end{gather*}
它们分别称为上Riemann积分和下Rinemann积分。

若上下积分相等,则称\(f\)在\([a,b]\)上是黎曼可积的,写作\(f\in \mathscr R\),即\(\sR\)表示黎曼可积函数的集合。我们把这个积分值表示为\[
\int_a^b f(x)~dx
\]

\end{definition}

\begin{remark}
    \(f\)是有界的,设\(m\leq f\leq M\),则\[
    m(b-a)\leq L(P,f)\leq U(P,f)\leq M(b-a)
    \]这说明上下和是有界的。从而上下Riemann积分对于有界函数(因为能总能在有界取值中取到确界值)总是良定义的,
\end{remark}

\begin{definition}
    设函数\(\alpha\)在\([a,b]\)上单调递增,显然\(\alpha\)有界。对应特定的划分\(P\),记\[
    \Delta \alpha_i=\alpha(x_i)-\alpha(x_{i-1})
    \]
    类似地,我们定义\begin{gather*}
    U(P,f,\alpha)=\sum_{i=1}^nM_i\Delta\alpha_i\\
    L(P,f,\alpha)=\sum_{i=1}^n m_i \Delta\alpha_i
    \end{gather*}
    以及\begin{gather*} 
\overline{\int_a^b} f \, d\alpha =   \inf U(P,f,\alpha)\\
\underline{\int_a^b} f \, d\alpha   =\sup L(P,f,\alpha)
    \end{gather*}
    当上下积分相等时,这个积分值记作\[
    \int_a^b f(x)~d\alpha(x)
    \]
    此时称\(f\)在\([a,b]\)上是Riemann-Stieltjes可积的,记作\(f\in \sR(\alpha)\)。
\end{definition}
以上给出了Riemann积分和它的推广Riemann-Stieltjes积分的定义。

\section{积分的性质}
\section{积分和微分}
\section{向量值函数的积分}
\section{可测曲线}
\end{document}